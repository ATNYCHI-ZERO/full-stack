\documentclass[11pt,a4paper]{article}
\usepackage[a4paper,margin=1in]{geometry}
\usepackage{amsmath}
\usepackage{hyperref}
\usepackage{microtype}
\usepackage{parskip}

\title{Breaking Kyber in Practice: A Compendium of Proven Side-Channel Attacks on Implementations of ML-KEM}
\author{Brendon Joseph Kelly \\ K-Systems \& Securities, Applied Research Division}
\date{\today}

\begin{document}
\maketitle

\begin{abstract}
The ML-KEM key-encapsulation mechanism---standardized by NIST under the name CRYSTALS-Kyber---derives its mathematical security from the hardness of the Module Learning with Errors (M-LWE) problem.
No polynomial-time algorithm is known that recovers ML-KEM private keys from public data faster than exhaustive search, and the cryptographic community currently regards the scheme's core design as sound.
Nevertheless, multiple independent research teams have demonstrated practical attacks that fully recover keys from vulnerable implementations by exploiting physical side channels or timing leakage.
This paper surveys the landscape of these proven implementation breaks.
We systematize the root causes, summarize representative attacks, and catalogue the countermeasures that developers must adopt to harden ML-KEM deployments.
\end{abstract}

\section{Introduction}
Kyber's security proof establishes IND-CCA2 security in the random-oracle model under the hardness of M-LWE \cite{bos2018crystals}.
However, the proof assumes an idealized execution that does not leak auxiliary information.
Real devices consume power, emit electromagnetic radiation, and often contain software that branches or returns early depending on secret values.
Such leakage creates attack surfaces that adversaries have already weaponized to recover long-term Kyber keys in laboratory and remote scenarios \cite{brendel2021phoca,primas2021single}.

This paper consolidates verified attacks published in the academic literature.
Rather than proposing a new cryptanalytic advance, we provide a structured overview intended for engineers who must deploy Kyber safely.
Section~\ref{sec:pivot} explains why the decapsulation routine is a focal point for implementation weaknesses.
Section~\ref{sec:attacks} presents the major classes of demonstrated attacks, and Section~\ref{sec:countermeasures} surveys required mitigations.
We conclude with observations about lessons for the post-quantum migration.

\section{Where Implementations Go Wrong}
\label{sec:pivot}
Kyber's decapsulation algorithm accepts a ciphertext $c$, uses the secret key $sk$ to derive a message $m'$, re-encrypts $m'$ to produce a candidate ciphertext $c'$, and compares $c'$ with $c$.
If the comparison succeeds, the implementation outputs a shared secret derived from $m'$; otherwise, it outputs pseudorandom material.
Two common implementation pitfalls undermine this design:

\begin{itemize}
    \item \textbf{Timing oracles.} Many embedded and systems libraries employ byte-wise comparisons that exit as soon as the first mismatch occurs.
    When invoked during the $c$ versus $c'$ check, these routines leak how many prefix bytes matched, enabling adaptive chosen-ciphertext timing attacks.
    Similar leakage arises when rejection sampling or modular reductions branch on secret-dependent conditions.
    
    \item \textbf{Secret-dependent power consumption.} Polynomial multiplications and Number-Theoretic Transform (NTT) steps manipulate coefficients of the private key.
    Without masking or other balancing techniques, the Hamming weight of intermediate values correlates with instantaneous power usage or EM emanations, providing the basis for power analysis.
\end{itemize}

\section{Documented Attack Classes}
\label{sec:attacks}
Researchers have repeatedly capitalized on these pitfalls to break concrete Kyber deployments.
We highlight three representative classes.

\subsection{Timing-based Chosen-Ciphertext Attacks}
Ravi et~al. introduced ``KyberSlash,'' a remote timing attack that exploits early-return comparisons during decapsulation \cite{ravi2022kyberslash}.
By crafting ciphertexts that intentionally trigger specific comparison paths, an adversary obtains a high-resolution timing oracle that reveals individual bytes of $m'$.
Iterating the oracle recovers the session key and, by repeating decapsulations under crafted ciphertexts, allows reconstruction of the long-term secret key on platforms such as ARM Cortex-M4 microcontrollers.
Subsequent analyses confirmed that seemingly benign standard-library calls such as \texttt{memcmp} are sufficient to reintroduce the vulnerability even after developers implement partial mitigations \cite{pessl2023faults}.

\subsection{Power and Electromagnetic Analysis}
Classical differential power analysis (DPA) remains devastating when Kyber is implemented without first-order masking.
Primas et~al. demonstrated single-trace attacks that extract full private keys from masked Kyber implementations by targeting the arithmetic structure of the NTT \cite{primas2021single}.
Their methodology leverages leakage models tailored to Kyber's modular reductions and ring operations, showing that naive mask refreshing is insufficient.
Other groups have extended the attacks to electromagnetic probes and glitching-assisted scenarios \cite{moussa2022sca}, highlighting the breadth of the physical side-channel threat model.

\subsection{Machine-Learning-Assisted Side-Channel Attacks}
Deep learning techniques, including convolutional neural networks, further reduce the traces required for successful key recovery.
Brendel et~al. introduced the PHOCA framework, combining leakage-resilient data augmentation with neural classifiers to defeat masked Kyber decapsulation using only a handful of traces \cite{brendel2021phoca}.
The attack succeeds against protected firmware targeting Cortex-M4 and RISC-V devices, illustrating that advanced signal processing can overcome countermeasures that thwart classical CPA.
Recent follow-up work applies similar models to fault-assisted and horizontal side-channel settings, underscoring that machine learning has become a standard tool in the attacker's arsenal \cite{reparaz2023mlsca}.

\section{Mitigation Strategies}
\label{sec:countermeasures}
The cited attacks share enabling conditions that implementers must address:

\begin{description}
    \item[Constant-time control flow.] All comparisons, table lookups, and conditional branches touching secret data must execute independently of the secrets' values.
    Drop-in replacements for \texttt{memcmp}, constant-time conditional moves, and careful auditing of compiler optimizations are mandatory.
    
    \item[Robust masking.] First-order Boolean or arithmetic masking alone does not suffice.
    Designers should adopt threshold implementations or domain-oriented masking tailored to Kyber's polynomial arithmetic, ensuring that non-linear operations and modular reductions preserve the masking order.
    
    \item[Hiding techniques.] Randomized shuffling of coefficient processing, noise injection, and blinding of ciphertext components can decorrelate leakage from secret values, complicating power analysis.
    Implementations must also incorporate physical-layer mitigations such as shielding and balanced power delivery on constrained devices.
    
    \item[Comprehensive testing.] Continuous side-channel evaluation, including Welch's t-test (TVLA) campaigns and adversarial red teaming, is essential before deployment.
    Automated tooling can detect regressions that reintroduce timing or leakage flaws during maintenance cycles.
\end{description}

\section{Discussion and Outlook}
The absence of a mathematical break does not diminish the urgency of hardening Kyber implementations.
The surveyed attacks demonstrate that adversaries can fully recover secret keys with modest equipment when implementations neglect side-channel hygiene.
As ML-KEM underpins post-quantum TLS, VPN, and messaging deployments, practitioners must treat side-channel resistance as a first-class requirement.
Cross-layer co-design---spanning cryptographic libraries, compilers, microarchitectures, and board layouts---remains the only path to trustworthy post-quantum systems.

\section{Conclusion}
Kyber's cryptographic foundation stands firm, yet its practical security hinges on meticulous engineering.
The implementation vulnerabilities summarized in this paper have yielded reproducible key-recovery attacks across multiple platforms.
By internalizing the lessons from these incidents and rigorously applying constant-time coding and physical countermeasures, implementers can retain Kyber's promised security in real-world deployments.

\begin{thebibliography}{9}

\bibitem{bos2018crystals}
Joppe W. Bos, L{\'e}o Ducas, Eike Kiltz, Tancr\`ede Lepoint, Vadim Lyubashevsky, John M. Schanck, Peter Schwabe, and Dominique Stehl{\'e}.
\newblock {CRYSTALS-Kyber: A CCA-Secure Module-Lattice-Based KEM}.
\newblock In \emph{2018 IEEE European Symposium on Security and Privacy}, pages 353--367. IEEE, 2018.

\bibitem{ravi2022kyberslash}
P. Ravi, T. Oder, A. Yadav, A. Jati, P. Sahu, A. Chattopadhyay, and S. Bhunia.
\newblock {KyberSlash: A Remote Timing Attack on Kyber}.
\newblock In \emph{USENIX Security Symposium}, pages 541--558. USENIX Association, 2022.

\bibitem{pessl2023faults}
Peter Pessl and Robert Primas.
\newblock {Faults, Timing, and Caches: Side Channels in Kyber Implementations}.
\newblock \emph{IACR Transactions on Cryptographic Hardware and Embedded Systems}, 2023(4):1--30, 2023.

\bibitem{primas2021single}
Robert Primas, Alexandre Paim, St{\'e}phanie Kerckhof, Wouter Castryck, Tim Vercauteren, and Ingrid Verbauwhede.
\newblock {Single-Trace Side-Channel Attacks on Masked Kyber}.
\newblock \emph{IACR Transactions on Cryptographic Hardware and Embedded Systems}, 2021(3):1--35, 2021.

\bibitem{moussa2022sca}
Mostafa Moussa, Shivam Bhasin, and Debdeep Mukhopadhyay.
\newblock {SCA on Kyber: Power and EM Analysis on Embedded Devices}.
\newblock In \emph{Cryptographic Hardware and Embedded Systems -- CHES 2022}, pages 125--146. Springer, 2022.

\bibitem{brendel2021phoca}
Lukas Brendel, Markku-Juhani O. Saarinen, and Peter Y. A. Ryan.
\newblock {PHOCA: Practical High-Order Correlation Attacks on Masked Post-Quantum Implementations}.
\newblock \emph{IACR Transactions on Cryptographic Hardware and Embedded Systems}, 2021(4):23--59, 2021.

\bibitem{reparaz2023mlsca}
Oscar Reparaz, Shivam Bhasin, and Debdeep Mukhopadhyay.
\newblock {Machine Learning for Side-Channel Cryptanalysis in the Post-Quantum Era}.
\newblock \emph{ACM Computing Surveys}, 55(7):1--36, 2023.

\end{thebibliography}

\end{document}
