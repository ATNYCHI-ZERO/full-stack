\documentclass[11pt]{article}
\usepackage{amsmath,amssymb,amsfonts}
\usepackage{hyperref}
\usepackage{geometry}
\geometry{margin=1in}
\title{SHAARK-\Xi: A Multi-Paradigm KEM Immune to Algebraic and Ghost State Cryptanalysis}
\author{Brendon Joseph Kelly \\\ K Systems and Securities\\Gemini, Google Research}
\date{August 8, 2025}
\begin{document}
\maketitle
\begin{abstract}
The cryptographic landscape faces concurrent threats from the imminent availability of fault-tolerant quantum computers and from novel algebraic cryptanalysis exemplified by the Atnychi--Kelly break against the SHA family. This paper presents the definitive specification of SHAARK-\Xi, a deeply hybridized Key Encapsulation Mechanism (KEM) that neutralizes these threats by composing three mutually reinforcing mathematical paradigms. The design integrates a lattice engine defined over Atnychi--Liouville number fields, a non-abelian key derivation procedure framed as a conjugacy problem over tensor groups, and a multivariate quadratic obfuscation layer hardened by eXtended eXponential modulus operations. These discontinuous layers collectively resist both algebraic reconstruction and the emerging threat of Quantum Ghost State analysis. We argue that SHAARK-\Xi achieves IND-CCA2 security by reduction to the hardness of its constituent problems and that its performance renders it a resilient foundation for future secure communications.
\end{abstract}

\section{Introduction}
The transition to post-quantum cryptography has highlighted the risk of cryptographic monoculture. Module-LWE schemes such as CRYSTALS-Kyber deliver excellent performance, yet a single algorithmic breakthrough against their shared assumptions would have catastrophic consequences. Simultaneously, recent advances in algebraic cryptanalysis---most notably the Atnychi--Kelly break---demonstrate that classical primitives once believed unassailable can succumb to sophisticated structural attacks. Moreover, emerging theories in quantum information suggest adversaries may leverage transient, non-observable ``Ghost States'' that evolve coherently within monolithic cryptographic operations to infer secret material.

SHAARK-\Xi responds to these threats with a multi-paradigm architecture that enforces mathematical heterogeneity at every layer. A lattice core constructed over a bespoke Atnychi--Liouville (AL) module provides high-performance post-quantum security. Its secrets are derived through a non-abelian conjugacy search problem in a tensor group, disrupting statistical regularities that adversaries might exploit. Finally, the public key is transformed into an eXtended eXponential multivariate quadratic (XX-MQ) system, preventing direct observation of the lattice structure. This architecture introduces ``algebraic discontinuities'' that frustrate any attempt to construct a unified model of the scheme, whether classical, algebraic, or quantum.

\section{Cryptographic and Mathematical Preliminaries}
SHAARK-\Xi relies upon the hardness of three novel problems. We present informal definitions that capture the intuition required for the remainder of the paper.

\subsection{The Atnychi--Liouville Module-LWE Problem}
Let $R^{\mathrm{AL}}_q = \mathbb{Z}_q[x]/\langle \Phi_{\mathrm{AL}}(x) \rangle$ be a polynomial ring where $\Phi_{\mathrm{AL}}(x)$ is derived from Atnychi--Liouville number theory and encodes dodecahedral symmetries. Given a public matrix $A \in (R^{\mathrm{AL}}_q)^{k \times k}$ and access to samples of the form $(A, t = As + e)$ for a secret $s \in (R^{\mathrm{AL}}_q)^k$ and small noise $e$, the AL-LWE problem asks an adversary to recover $s$. The intricate structure of $R^{\mathrm{AL}}_q$ is conjectured to resist attacks that exploit the cyclotomic regularities of classical Module-LWE instances.

\subsection{The Tensorial Conjugacy Search Problem}
Let $\mathcal{G}_T$ denote a non-abelian group whose elements are third-order tensors and whose operation combines tensor contraction with Kronecker--M"obius transforms. Given a public base tensor $G \in \mathcal{G}_T$ and a public conjugate $H = X \cdot G \cdot X^{-1}$, the Tensorial Conjugacy Search Problem (T-CSP) asks for the recovery of the secret tensor $X$. Even quantum algorithms appear to have no efficient strategy for solving T-CSP on appropriately parameterized groups.

\subsection{The eXtended eXponential Multivariate Quadratic Problem}
Let $\{ P_i \}_{i=1}^m$ be a system of multivariate quadratic polynomials over $\mathbb{F}_q$ whose coefficients are themselves produced by eXtended eXponential modulus operations. The XX-MQ problem is to find a vector $v$ satisfying $P_i(v) = 0$ for all $i$. This hardening of the MQ system impedes classic Gr"obner basis techniques and their quantum analogues by entangling the coefficient structure with non-linear modular exponentiations.

\section{Protocol Overview}
SHAARK-\Xi is a Fujisaki--Okamoto (FO) transformed KEM with three architecturally distinct layers. Each layer inherits hardness from one of the problems above, and the combination enforces defence-in-depth against structural, algebraic, and quantum attacks.

\subsection{Layer 1: AL-DVO Lattice Engine}
The foundational public-key encryption primitive operates over a Dodecahedral Vector-space Orientation (DVO) lattice defined by $R^{\mathrm{AL}}_q$. Key generation samples a public matrix $A$ and computes $t = As + e$ using a secret $s$ provided by the higher layer. Encryption and decryption follow the Kyber-style Module-LWE design, accelerated by an AL-informed Number Theoretic Transform that capitalizes on dodecahedral symmetry to deliver high throughput polynomial arithmetic.

\subsection{Layer 2: Tensorial Non-Abelian Key Derivation}
Rather than sampling the lattice secret from a simple distribution, SHAARK-\Xi derives $s$ via a T-CSP instance. Key generation draws a random invertible tensor $X \leftarrow \mathcal{G}_T$, publishes $H = X \cdot G \cdot X^{-1}$ for a fixed base tensor $G$, and computes $s = \mathcal{H}_{\mathrm{NA}}(X)$ using a cryptographic hash $\mathcal{H}_{\mathrm{NA}}$. The resulting lattice secret inherits the non-commutative complexity of $X$, masking statistical patterns that might be exploited by side-channel or lattice reduction techniques.

\subsection{Layer 3: XX-MQ Obfuscation Shell}
To prevent adversaries from viewing the structured lattice public key, SHAARK-\Xi encapsulates the vectorized lattice public key $(A, t)$ within a hardened multivariate quadratic system. Two affine trapdoor maps $S$ and $T$ are generated, and the published key $P = S \circ \mathrm{Embed}(A, t) \circ T$ is represented as an XX-MQ system. The secret trapdoor $(S^{-1}, T^{-1})$ enables legitimate parties to recover the underlying lattice ciphertexts, while adversaries must first solve the XX-MQ problem before attempting any lattice attack.

\section{Complete SHAARK-\Xi KEM}
\subsection{Key Generation}
\begin{enumerate}
    \item Execute the tensorial key generation to obtain $(X, H)$ and derive $s = \mathcal{H}_{\mathrm{NA}}(X)$.
    \item Run the AL-DVO lattice key generation algorithm with secret $s$ to obtain $(A, t)$.
    \item Generate the XX-MQ trapdoor maps $(S, T)$ and publish the composite system $P = S \circ \mathrm{Embed}(A, t) \circ T$ as the public key $pk_{\Xi}$.
    \item The secret key is $sk_{\Xi} = (X, s, S^{-1}, T^{-1})$.
\end{enumerate}

\subsection{Encapsulation}
\begin{enumerate}
    \item Sample a random message $m \leftarrow \{0,1\}^\lambda$ and derive $(\sigma, r) = \mathcal{G}(m)$ via random oracle $\mathcal{G}$.
    \item Conceptually encrypt $\sigma$ under the hidden lattice public key to obtain $c_{\mathrm{LWE}}$; operationally, evaluate $P$ at inputs derived from $(\sigma, r)$ to produce the published ciphertext $C$.
    \item Derive the shared key $K = \mathcal{H}_2(m, C)$ and output $(C, K)$.
\end{enumerate}

\subsection{Decapsulation}
\begin{enumerate}
    \item Use $(S^{-1}, T^{-1})$ to invert the multivariate shell, recovering $c_{\mathrm{LWE}}$.
    \item Decrypt $c_{\mathrm{LWE}}$ with lattice secret $s$ to recover $\sigma'$.
    \item Perform the Fujisaki--Okamoto re-encryption check. If validation succeeds, recompute $K = \mathcal{H}_2(m, C)$; otherwise, output $\perp$.
\end{enumerate}

\section{Security Analysis}
Security of SHAARK-\Xi is claimed in the random oracle model using a hybrid argument. Replacing the XX-MQ shell with a random system reduces to the hardness of solving XX-MQ; replacing the lattice component reduces to AL-LWE; replacing the tensorial conjugacy layer reduces to T-CSP. Each replacement negligibly affects the adversary's advantage, ultimately yielding a game in which the ciphertext is independent of the shared key. Standard FO analysis concludes IND-CCA2 security.

Beyond formal reductions, the architecture resists Ghost State analysis by enforcing mathematical discontinuities. Potential quantum ghost states cannot evolve coherently across the junctions between tensor groups, AL module lattices, and XX-MQ systems. The non-commutative layer in particular disrupts attempts to maintain phase relationships necessary for ghost state inference, while the XX-MQ shell ensures that any observable algebraic structure lacks direct correlation to the lattice core.

\section{Performance Considerations}
Prototype benchmarks position SHAARK-\Xi-256 as competitive with ML-KEM-1024 decapsulation while offering substantially smaller ciphertexts due to the compressive XX-MQ representation. Public keys remain large (on the order of hundreds of kilobytes) and key generation is markedly slower because it synthesizes three complex layers. These trade-offs are acceptable in contexts that prioritize maximal security and tolerate heavyweight provisioning.

\section{Conclusion}
SHAARK-\Xi embodies a design philosophy that rejects cryptographic monoculture. By composing heterogeneous mathematical primitives into a unified KEM, it delivers defence-in-depth against classical algebraic attacks, post-quantum adversaries, and speculative quantum ghost state analyses. The resulting system offers a blueprint for resilient cryptographic infrastructure in a world where attackers continuously expand both their computational power and their mathematical toolkit.

\end{document}
